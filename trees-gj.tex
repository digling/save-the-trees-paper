\clubpenalty100000000
\widowpenalty100000000
\documentclass[svgnames,12pt]{scrartcl}
\usepackage{cgloss4e} 
\usepackage{xltxtra,polyglossia}
\usepackage{graphicx}
\usepackage[round]{natbib}
\usepackage[table]{xcolor}
\usepackage{booktabs}
\usepackage{color, colortbl, multirow}
\usepackage{hyperref} 
\hypersetup{bookmarks=false,bookmarksnumbered,bookmarksopenlevel=5,bookmarksdepth=5,xetex,colorlinks=true,linkcolor=blue,citecolor=blue}

\setmainfont{Times New Roman}
\usepackage[all]{hypcap}
\usepackage{memhfixc}
\usepackage{tikz}
\usetikzlibrary{trees,arrows,arrows.meta}
\usepackage{gb4e}  
\setmainlanguage{english}
\fontfamily{ptm}\selectfont

\setcitestyle{aysep={},citesep={,},notesep={:}}
%\newfontfamily\hana{HAN NOM A}
%\newcommand\Chinese[1]{{\hana #1}}
\newcommand\Comment[1]{\textcolor{red}{#1}}
\newfontfamily\phon[Mapping=tex-text,Ligatures=Common,Scale=MatchLowercase]{Charis SIL} 
\newcommand{\ipa}[1]{\textbf{{\phon\mbox{#1}}}} 
\newcommand{\grise}[1]{\cellcolor{lightgray}\textbf{#1}}
\newfontfamily\cn{HAN NOM A}%pour le chinois
\newcommand{\zh}[1]{{\cn#1}}

\title{Save the Trees: Why we need tree models in linguistic reconstruction}
\author{Guillaume Jacques and Johann-Mattis List}  
%acknowledgements: Juho Pystynen, Mikhail Zhivlov, Nathan Hill, Mathieu Ségui, John Cowan, Simon Greenhill, Dmitry Nikolaev, Yoram Meroz, Tiago Tresoldi, Rémy Viredaz, Martin Kümmel
\begin{document}
\maketitle


\subsection{The Problem of Identifying Lexical Innovations} \label{sec:ILS}
In order to identify inherited lexical innovations and distinguish them from recent borrowings, the method of historical glottometry uses a fairly uncontroversial criterion: Etyma whose reflexes follow regular sound correspondences are considered to be inherited \citep[176-8]{Francois2015}.
Thus, whenever a common proto-form can be postulated for a particular set of words across several languages (which can thus be derived from this proto-form by the mechanical application of regular sound changes), it is considered in this model to be part of the inherited vocabulary, and can be
used, if applicable, as a common innovation.

This approach however neglects an important factor: while regular sound correspondences is a necessary condition for analyzing forms in related languages as cognates, i.e. originating from the same etymon in their common ancestor,\footnote{Note however that cognacy is a more complex concept that is usually believed (\citealt{list16cognacy}), and that even forms originating from exactly the same etymon in the proto-language may present irregular correspondences due to analogy.} it is not a \textbf{sufficient} condition due to the existence of \textbf{undetectable borrowings} and
\textbf{nativized loanwords}.  
 

\subsubsection{Undetectable Borrowings}
Sound changes are not always informative enough to allow the researcher to discriminate between inherited word and borrowing. When a form contains phonemes that remained unchanged, or nearly unchanged, from the proto-language in all daughter languages (because no sound change, or only trivial changes, affected them), there is no way to know whether it was inherited from the proto-language or whether it was borrowed at a later stage. 

This type of situation is by no means exceptional, and can be found in various language families. We present here two examples of borrowings undetectable by phonology alone: `aluminum' in Tibetan languages and `pig' in some Algonquian languages. 
Amdo Tibetan \ipa{hajaŋ} `aluminum' and Lhasa \ipa{hájã} `aluminum' look like they regularly originate from a Common Tibetan form *\ipa{ha.jaŋ}.\footnote{In Amdo Tibetan, Common Tibetan \ipa{h-}, \ipa{j-}, \ipa{-a} and \ipa{-aŋ} remain unchanged (\citealt{gong16amdo}). In Lhasa Tibetan, two sound changes relevant to this form occurred: a phonological high tone developped with the initial \ipa{h-}, and \ipa{-aŋ} became nasalized \ipa{ã}.} This is of course impossible for obvious historical reasons, as aluminum came into use in Tibetan areas in the twentieth century, at a time when Amdo Tibetan and Lhasa Tibetan were already mutually unintelligible. This word is generally explained (Gong Xun, p.c.) as an abbreviated form of \ipa{ha.tɕaŋ jaŋ.po} `very light', but this etymology is not transparent to native speakers of either Amdo or Lhasa Tibetan. This word has been coined only once,\footnote{We are not aware of a detailed historical research on the history of this particular word, but in any case it matters little for our demonstration whether it was first coined in Central Tibetan or in Amdo.} and was then borrowed into other Tibetan languages\footnote{In some Tibetan languages such as Cone \ipa{hæ̀jãː}, \citet[306]{jacques14cone}, there is clear evidence that the word is borrowed from Amdo Tibetan and is not native (otherwise $\dagger$\ipa{hæ̀jaː} would have been expected). } and neighboring minority languages under Tibetan influence (as for instance Japhug \ipa{χajaŋ} `aluminum').
In this case a phonetic borrowing from Amdo \ipa{hajaŋ} could only yield Lhasa \ipa{hájã}, since \ipa{h-} only occurs in high tone in Lhasa, and since final \ipa{-ŋ} has been transphonologized as vowel nasality.\footnote{Likewise, in the case of borrowing from Lhasa into Amdo, the rhyme \ipa{-aŋ} would be the only reasonable match for Lhasa \ipa{-ã}.} 

Several Algonquian languages, share a word for `pig' (Fox \ipa{koohkooša}, Miami \ipa{koohkooša} and Cree \ipa{kôhkôs}) ultimately of Dutch origin (\citealt{goddard74dutch}, \citealt{costa13borrowing}). \citet[266]{hockett57k} pointed out that these forms must be considered to be loanwords `because of the clearly post-Columbian meaning; but if we did not have the extralinguistic information the agreement in shape (apart from M[enominee]) would lead us to reconstruct a [Proto-Central-Algonquian] prototype.' The forms from these three languages could be regularly derived from Proto-Algonquian *\ipa{koohkooša}, a reconstruction identical to the attested Fox and Miami forms.
%Ojibwe gookoosh

%Semitic languages abound in common vocabulary which presents the same correspondences as inherited vocabulary, but which was diffused after the breakup of the family. For instance, from Biblical Hebrew \ipa{hêkāl} `palace' and Arabic \ipa{haykal} `palace', it would be possible to reconstruct a Proto-Semitic etymon *\ipa{haykal(u)}; it is however well-known that these words originate from Sumerian \ipa{é.gal} `palace', probably through Akkadian \ipa{ekallum}, and that borrowing from Akkadian took place at a time when the ancestors of Hebrew and Arabic respectively were already distinct languages.

Undetectable borrowings are also a pervasive phenomenon in Pama-Nyungan, where with a few exceptions such as the Arandic and Paman groups, most languages present too few phonological innovations to allow easy discrimination for loanwords from cognates (\citealt[46]{koch04method}).

The same situation can be observed even if later sound changes apply to both borrowings and inherited words. Whenever borrowing takes place after the separation of two languages, but before any diagnostic sound change occurred in either the donor or the receiver language, or if the donor and the receiver languages underwent identical sounds changes up to the stage when the borrowings occurred, phonology alone is not a sufficient criterion to distinguish between inherited words and loanwords. 
 
A classical case is that of Persian borrowings in Armenian. As
\citet[16-17]{huebschmann97armenische} put it, `in isolated cases, the Iranian and the genuine Armenian forms match each other phonetically, and the question whether borrowing [or common inheritance] has to be assumed must be decided from a non-linguistic point of view.'\footnote{Our
translation, original text: `In einzelnen Fällen kann allerdings das persische und echt armenische Wort sich lautlich decken und die Frage, ob Entlehnung anzunehmen ist oder nicht, muss dann nach andern als sprachlichen Gesichtspunkten entschieden werden.'} Table \ref{tab:armenian} presents a non-exhaustive list of such words. In these words, 
 
\begin{table}[h]
\caption{Armenian words which cannot be conclusively demonstrated to be either borrowings from Iranian or inherited words from a phonetic point of view    } \centering \label{tab:armenian}
\begin{tabular}{lllllll}
\toprule 
Armenian & Meaning & Indo-Iranian & Reference \\
\midrule 
\ipa{naw}& boat & Skt. \ipa{nau-} & \citet[16-17;201]{huebschmann97armenische},\\
&&& \citet[466;715]{martirosyan10etymological} \\
\midrule 
\ipa{mēg}& mist & Skt. \ipa{megha-},  & \citet[474]{huebschmann97armenische},\\
&&Avestan \ipa{maēɣa-}& \citet[466;715]{martirosyan10etymological} \\
\midrule 
\ipa{mēz}& urine & Skt. \ipa{meha-} & \citet[474]{huebschmann97armenische},\\
&&& \citet[466;715]{martirosyan10etymological} \\
\midrule 
\ipa{sar}& head & Skt. \ipa{śiras-} & \citet[236;489]{huebschmann97armenische},\\
&&Y.Avestan \ipa{sarah-}& \citet[571]{martirosyan10etymological} \\
\midrule 
\ipa{ayrem}& burn & Skt. \ipa{edh-} & \citet[418]{huebschmann97armenische},\\
&&& \citet[145]{martzloff16geri} \\
\bottomrule
\end{tabular}
\end{table}
 
The Armenian case shows that undetectable loans are not restricted to cases like those studied above, when a particular word only contains segments which have not been affected by sound changes from the proto-language to all its daughter languages. Undetectable loans are possible when a particular word is borrowed before any sound change which could affect its phonetic material occurred in either the giver or recipient language (or if both languages have identical sound changes for words of this particular shape), even if numerous sound changes occurred \textit{after} borrowing took place. It is possible that post-borrowing sound changes even remove phonetic clues which could have allowed to distinguish between loanwords and inherited words.

What has been illustrated above can be seen as clear evidence that undetectable borrowings can occur even when two language varieties are mutually unintelligible. Neglecting the distinction between inherited words and undetectable borrowings, as in the approach propagated in historical glottometry, amounts to losing crucial historical information, and it does not seem justified to blame the family tree model for an insufficiency of our linguistic reconstruction methodology.

\subsubsection{Nativization of Loanwords}
In the previous section, we discussed cases when borrowing took place before diagnostic sound changes, thus making it impossible to effectively use sound changes to distinguish between loanwords and inherited words. There is however evidence that even when diagnostic sound changes exist, they may not always be an absolutely reliable criterion.

When a particular language contains a sizeable layer of borrowings from another language, bilingual speakers can develop an intuition of the phonological correspondences between the two languages, and apply these correspondences to newly borrowed words, a phenomenon known as loan nativization.

The best documented case of loan nativization is that between Saami and Finnish (the following discussion is based on \citealt{aikio06nativization}). Finnish and Saami are only remotely related within the Finno-Ugric branch of Uralic, but  Saami has borrowed a considerable quantity of vocabulary from Finnish, some at a stage before most characteristic sound changes had taken place, other more recently. Table \ref{tab:native} presents examples of cognates between Finnish and Northern Saami illustrating some recurrent vowel and consonant correspondences.

\begin{table}[h]
\caption{Examples of sound correspondences in inherited words between Finnish and Northern Saami (data from \citealt[27]{aikio06nativization})} \centering \label{tab:native}
\begin{tabular}{lllll}
\toprule
Finnish & Northern Saami & Proto-Finno-Ugric & Meaning \\
\midrule
\ipa{käsi} & \ipa{giehta} & *\ipa{käti} & `hand' \\
\ipa{nimi} & \ipa{namma} & *\ipa{nimi} & `name' \\
\ipa{kala} & \ipa{guolli} & *\ipa{kala} & `fish' \\
\ipa{muna} & \ipa{monni} & *\ipa{muna} & `egg' \\
\bottomrule
\end{tabular}
\end{table}

The correspondence of final \ipa{-a} to \ipa{-i} and final \ipa{-i} to \ipa{-a} in disyllabic words found in the native vocabulary, as illustrated by the data in Table \ref{tab:native}, is also observed in Saami words borrowed from Finnish, including recent borrowings, such as \ipa{mearka} from \ipa{merkki} `sign, mark' and \ipa{báhppa} from \ipa{pappi} `priest' (from Common Slavic *\ipa{păpъ}, itself of Greek origin), even though the sound change from proto-Uralic to Saami leading to the correspondence \ipa{-a} : \ipa{-i} had already taken place at the time of contact. These correspondences are pervasive even in the most recent borrowings, to the extent that according to \citet[36]{aikio06nativization} `examples of phonetically unmarked substitutions of the type F[innish] \ipa{-i} > Saa[mi] \ipa{-i} and F[innish] \ipa{-a} > Saa[mi] \ipa{-a} are practically nonexistent, young borrowings included.'

In cases such as \ipa{báhppa} `priest', the vowel correspondence in the first syllable \ipa{á} : \ipa{a} betrays its origin as a loanword, as the expected correspondence for a native word would be \ipa{uo} : \ipa{a} as in the word `fish' in Table \ref{tab:native} (\citealt[35]{aikio06nativization} notes that this correspondence is never found in borrowed words).

However, there are cases where recent loanwords from Finnish in Saami present correspondences indistinguishable from those of the inherited lexicon, as \ipa{barta} `cabin' from Finnish \ipa{pirtti}, itself from dialectal Russian \ipa{перть} `a type of cabin', which show the same \ipa{CiCi} : \ipa{CaCa} vowel correspondence as the word `name' in table \ref{tab:native}. Here again, the foreign origin of this word is a clear indication that \ipa{barta} `cabin' cannot have undergone the series of regular sound changes leading from proto-Finno-Ugric *\ipa{CiCi} to Saami \ipa{CaCa}, and that instead the common vowel correspondence \ipa{CiCi} : \ipa{CaCa} was applied to Finnish \ipa{pirtti}.
 
Loan nativization can also occur between genetically unrelated languages. A clear example is provided by the case of Basque and Spanish (\citealt[53-54]{trask00chronology}, \citealt[21-3]{aikio06nativization}). 
A recurrent correspondence between Spanish and Basque is word-final \ipa{-ón} to \ipa{-oi}. Early Romance *\ipa{-one} (from Latin \ipa{-onem}) yields Spanish \ipa{-ón}. In Early Romance borrowings into Basque, however, this ending undergoes the regular loss of intervocalic *\ipa{-n-} (a Basque-internal sound change), and yields *\ipa{-one} $\rightarrow$ *\ipa{-oe} $\rightarrow$ \ipa{-oi}. An example of this correspondence is provided by Spanish \ipa{razón} and Basque \ipa{arrazoi} `reason' both from Early Romance *\ipa{ratsone} (from the Latin accusative form $\leftarrow$ \ipa{ratiōnem}).
This common correspondence has, however, been recently applied to recent borrowings from Spanish such as \ipa{kamioi} `truck' and \ipa{abioi} `plane' (from \ipa{camión} and \ipa{avión}). This adaptation has no phonetical motivation, since word-final \ipa{-on} is attested in Basque, and can only be accounted for as overapplication of the \ipa{-oi} : \ipa{-ón} correspondence.
 
Nativization of loanwords is still a poorly investigated phenomenon and can only be detected in language groups whose historical phonology is already very well understood. While it has not yet been documented as clearly as in Saami and Basque, there is no reason to believe that this phenomenon is rare crosslinguistically. Its existence implies that sound laws cannot be used as an absolute criterion for distinguishing between inherited and borrowed common vocabulary (and thus between true shared innovations and post-innovation borrowings).

 
\subsection{The Benefit of Trees in Language Comparison}
%\textcolor{red}{Write briefly about the benefit of trees: they give us polarity where we don't find it when we lack information on polarity of processes, 
Trees have several distinct advantages over more complex types of network, which make the tree model preferable in the absence of evidence of its inapplicability (on which see section \ref{sec:limits}).

\subsubsection{Parallel innovations}
Trees can be used to detect cases of parallel innovations or features spread through contact. A typical example of such situation in provided by Semitic. As shown by Table \ref{tab:protosem}, Hebrew and Akkadian share no less than four common innovative sound changes in the evolution of their consonantal systems: 
\begin{itemize}
\item \ipa{*θ} $\rightarrow$ \ipa{ʃ} (merging with \ipa{*ʃ})
\item \ipa{*ð} $\rightarrow$ \ipa{z} (merging with \ipa{*s})
\item \ipa{*θ'} $\rightarrow$ \ipa{s'} (merging with \ipa{*s'})
\item \ipa{*ɬ'} $\rightarrow$ \ipa{s'} (merging with \ipa{*s'})
\end{itemize}

\begin{table}[h]
\caption{Reflexes of proto-Semitic coronals in a selected set of Semitic languages (\citealt{huehnergard97}; innovative features shared by Akkadian and another language are indicated in grey)} \label{tab:protosem} \centering
\begin{tabular}{lllllll}
\toprule
Proto-Semitic & Akkadian & Hebrew & Biblical Aramaic & Arabic \\
\midrule
\ipa{*t} & \ipa{t} & \ipa{t} & \ipa{t} & \ipa{t} \\
\ipa{*d} & \ipa{d} & \ipa{d} & \ipa{d} & \ipa{d} \\
\ipa{*θ} & \ipa{ʃ} \grise{}& \ipa{ʃ} \grise{}& \ipa{t} & \ipa{θ} \\
\ipa{*ð} & \ipa{z} \grise{}& \ipa{z} \grise{}& \ipa{d} & \ipa{ð} \\
\ipa{*s} & \ipa{s} & \ipa{s} & \ipa{s} & \ipa{ʃ} &\\
\ipa{*z} & \ipa{z} & \ipa{z} & \ipa{z} & \ipa{z} \\
\ipa{*ʃ} & \ipa{ʃ} & \ipa{ʃ} & \ipa{ʃ} & \ipa{s} &\\
\ipa{*ɬ} & \ipa{ʃ} \grise{}& \ipa{ɬ} & \ipa{ɬ} &  \ipa{ʃ} \grise{}\\
\ipa{*t} & \ipa{t'} & \ipa{t'} & \ipa{t'} & \ipa{tˤ} \\
\ipa{*θ'} & \ipa{s'} \grise{}& \ipa{s'} \grise{}& \ipa{t'} & \ipa{ðˤ} \\
\ipa{*s'} & \ipa{s'}  & \ipa{s'} & \ipa{s'} & \ipa{sˤ} &\\
\ipa{*ɬ'} & \ipa{s'} \grise{}& \ipa{s'} \grise{}& \ipa{ʕ} &  \ipa{dˤ} \\
\bottomrule
\end{tabular}
\end{table}

While phonology could seem at first glance to support grouping Akkadian and Hebrew together in a group excluding Aramaic and Arabic, the bulk of morphological and lexical innovations incontrovertibly support that Akkadian is the first branch of the family, and that Aramaic and Hebrew are closer to each other than either is to Arabic  (see for instance \citealt{hetzron76two, huehnergard06protosemitic}; bayesian phylogenetic analyses have been proposed for Semitic and confirm this insight, see for instance \citealt{nicholls11semitic}). 

The tree here provide us with near-certainty that the innovative features shared by Hebrew and Akkadian are either parallel innovations or isoglosses transmitted through contact, and cannot be common innovations of these two languages.

\begin{figure}[h]
\caption{A simplified Stammbaum of Semitic languages} \label{fig:semitic}   \centering
  \begin{tikzpicture}
   \node (A) at (0,0) {Proto-Semitic};
   \node (B) at (-3,-2) {Akkadian};
   \node (C) at (3,-2) {West Semitic};
   \node (D) at (0,-4) {North-West Semitic};
   \node (E) at (-2,-6) {Hebrew};
   \node (F) at (2,-6) {Aramaic};   
   \node (G) at (8,-6) {Arabic};   
\tikzstyle{sur}=[-,very thick,>=latex]
\draw[sur] (A)--(B);
\draw[sur] (A)--(C);
\draw[sur] (C)--(D);
\draw[sur] (D)--(E);
\draw[sur] (D)--(F);
\draw[sur] (C)--(G);
\end{tikzpicture}
\end{figure}

\subsubsection{Reconstruction of the Ursprache}
Trees can be used to determine which features are reconstructible to the Ursprache, and which are more likely to be later innovations. To illustrate this application of tree models, let us take the case of Semitic  prepositions. 

Akkadian differs from the rest of the family in that its spatial prepositions are \ipa{in} and \ipa{ana}, while the other languages have forms going back to \ipa{*l-} and \ipa{*b-}. Geez (an Ethio-Semitic language, belonging to a sub-branch of West Semitic) however has a cognate of Akkadian \ipa{in}, the preposition \ipa{ən} which appears in some expressions (\citealt[16]{huehnergard06protosemitic}, \citealt[119]{kogan15semitic}), and Akkadian does have a frozen trace of the preposition \ipa{*b-} (\citealt[45-6]{rubin05semitic}). Since none of the four prepositions are the result of recent and obvious grammaticalization, there is no way without the tree model to decide which should be reconstructed to proto-Semitic and which should not. 

With  the Stammbaum in Fig. \ref{fig:semitic}, however, we know that since the prepositions  \ipa{*inV} and  \ipa{*b-} are attested (even as traces) in both Akkadian and West-Semitic and are not recently grammaticalized, they can be safely reconstructed to proto-Semitic.

\subsubsection{Directionality of change}
As a by-product of the reconstruction of particular features to the proto-language, trees can be used to determine the directionality of changes in ambiguous cases. While the directionality can sometimes be determined using the body of attested knowledge on sound (in particular \citealt{kuemmel07wandel}) or semantic changes (see for instance \citealt{urban11semantic}), there are still many isoglosses, in particular in inflexional morphology, whose interpretation as innovations or retentions is nearly impossible by direct comparison between languages. 
%On the other hand a tree based on a particular set of features (for instance, isoglosses in the vocabulary) can be meaningfully used to study the directionality of changes in other domains (for instance morphology) without risk of circularity.

As an example of how trees can be useful to determine the directionality of a semantic change, let us examine the root *\ipa{ʔmr} in Semitic (\citealt[233;331;544]{kogan15semitic}). This root is attested in various languages with a slightly different meaning; Table \ref{tab:amr} provides its reflexes in several languages. The meaning of this root is highly divergent across the languages; it is a perception verb (`see',  look at') in some and a verb of speech (`say',  command') in others. It is not obvious at first glance which is the primitive meaning.

\begin{table}[h]
\caption{Reflexes of the root *\ipa{ʔmr} in several Semitic languages (\citealt[233;331;544]{kogan15semitic})} \centering \label{tab:amr}
\begin{tabular}{llll}
\toprule
Language & Reflex & Meaning \\
\midrule
Akkadian & \ipa{amārum} & to see \\
Hebrew & \ipa{ʔāmar} & to say, to declare, to command \\
Ugaritic & \ipa{ʔmr} & to say; to look at \\
Arabic & \ipa{ʔamara} & to order \\
\bottomrule
\end{tabular}
\end{table}
However, the tree provides  a scenario for how the meaning of this root evolved across the family. The use of this root as a perception verb is found in both Akkadian and North-Weste Semitic (Akkadian), and is thus most likely to be the archaic meaning. Ugaritic, where the root means both `to look at' and `say', represents an intermediate stage, where both meanings were still in competition (this may be a preservation of the proto-West-Semitic stage). In Hebrew and Arabic, the use of this root as a perception verb has disappeared, and Arabic has further narrowed down its meaning to `command'.

The pathway of semantic change (\ref{ex:amr}) is a possible account of the evolution of the meaning of this root in Semitic that is compatible with the tree in Figure \ref{fig:semitic}.

\begin{exe}
\ex \label{ex:amr}
\glt `see' $\rightarrow$ `look at' $\rightarrow$ `address' $\rightarrow$ `say' $\rightarrow$ `command'
\end{exe}

In this particular case, the tree model not only help solving an ambiguous question in proto-Semitic reconstruction, it provides evidence for a semantic change that might not otherwise not have been clearly attested.

 
 \subsubsection{Language change and migration history}
Trees can be used to make sense of population prehistory and can help to better compare linguistic and archaeological evidence. Clues can be obtained regarding the history and the spread of a language family using the vocabulary reconstructible for particular nodes. 

For instance, the presence of a reconstructible etymon *\ipa{kasp-}  `silver'  in Akkadian (\ipa{kaspum}), Ugaritic \ipa{ksp} and Hebrew (\ipa{késeɸ}) among other languages (\citealt[14-6]{huehnergard12ugaritic}), suggests that silver smelting could have been possibly known to the speakers of proto-Semitic, an idea supported by the evidence of cupellation in Syria as early as the 4th millenium BC  (\citealt{pernicka98silver}). 

Other metals however are only reconstructible to lower branches of the family; for instance `iron' is not earlier than proto-Cananean (*\ipa{barðill-}, Hebrew \ipa{barzel}, cf \citealt[287]{kogan15semitic}; similar forms in other languages such as Akkadian \ipa{parzillum} `iron' do not follow the regular correspondences and cannot be cognate), an observation compatible with the much later spread of iron technology (\citealt{yahalom15iron}).

%Similarly, an etymon *\ipa{paraʃ-} `horse' is potentially reconstructible to proto-West Semitic (but not proto-Semitic, since it is absent from Akkadian, see \citealt[94]{kogan15semitic})

Of course, as shown in section \ref{sec:ILS}, words that are compatible with the sound laws of inherited vocabulary may nevertheless be diffused by contact (especially a form like*\ipa{kasp-} which remained unchanged in most of the ancient attested languages), and thus `linguistic paleontology' should be used with caution.

 \subsubsection{Rate of change}

the branch lengths   


\subsection{The Limits of the Tree model} \label{sec:limits}
While the tree model has undeniable advantages and remains the most powerful model for understanding the vertical history of most languages, there undoubtedly remains a residue of cases where it is not applicable even taking ILS into account, when one languages results from the merger of two previously unintelligible languages (whether or not the two varieties are demonstrably related or not).

The clearest and best documented example of this type is Michif, a contact language based on Canadian French and Plains Cree (\citealt{bakker97michif}). Example (1) (taken from \citealt{antonov15michif}) illustrates the main features of this languages (elements from French are in bold, and those from Cree in italics). Nearly all verbs and verbal morphology come from Plains Cree, except the verbs `to be' and `have' which are from French with their complete irregular paradigms (including French tense categories, as shown by example 2). Most nouns and adjectives come from French. Some determiners are from French (the articles) but the demonstrative are from Cree, and nouns can take the Cree obviative suffix \ipa{-(w)a} and some nouns are compatible with possessive prefixes (like \ipa{o-} below). 

\begin{exe}
\ex \gll \textit{o}-\textbf{pâpa}-\textit{wa} \textit{êtikwenn} \textit{kî-wîkimê-yiw} \textit{onhin} \textbf{la} \textbf{fâm}-\textit{a} \\
3-father-\textsc{obv} apparently \textsc{pst}-marry-\textsc{3obv$\rightarrow$3prox} \textsc{this:an:obv} 
\textsc{def:fem:sg} woman-\textsc{obv}\\
\glt ‘Her father apparently married that woman...’ (1:8-9)
\end{exe}

\begin{exe}
\ex \gll \textbf{stit=enn} \textbf{pchit} \textbf{orfelin} \\
\textsc{be:3sg:pst=indef:f:sg} little orphan \\
\glt ‘She was a little orphan’ (1:2)
\end{exe} 

The descent of a language like Michif, a potentially also that of less extreme contact languages, cannot in be represented by the tree model, as it requires two roots (from languages belonging to unrelated families). A more complex type of network, such as XXX, is necessary to represent a language of the type, and the near-perfect division of the French and the Cree components of this language may allow for a meaningful representation of the nature of language mixture.

The applicability of the tree model on a global scale crucially depends on the rarity of languages like Michif. If, as the data available to us seem to show, this language is truly exceptional (because its genesis occurred in a very special setting that is unlikely to have existed at earlier stages of history), there are neither practical nor theoretical obstacles against accepting the tree model to represent the vertical descent of languages.

\bibliographystyle{unified}
\bibliography{bibliography,example,bibliogj}

\end{document}

